\chapter{Introduction}
\label{chap:Introduction}
One of the most common and fundamental necessity of life is food. It is not only
about the joy of eating, it also has an intimate relationship with the important
factors of human life such as health, dietary, entertainment. Many people who
are busy in their work place, have to take their breakfast and lunch at
cafeteria or Restaurant. In that case, time, quality of food and appropriate
food are very important fact. Those people needs some extra facilities for
planning their menu and save their time for searching food and also spend less
time in the queue for food. It is obvious that, good and enjoyable food in a
short time would improve the quality of life\footnote{Some ideas \& contents
have taken from Silvia Torsi, Researcher, University of Trento.}.

``Smart Cafeteria'' is a part of Smart Campus project\footnote{This work is a
part of \href{http://www.smartcampuslab.it/}{Smart Campus} project funded by
\href{http://www.trentorise.eu/}{Trento RISE}.} funded by Trento RISE. The goal
of the project is to provide advanced Information and Communication Technology
(ICT) solutions to all people  involved in a University campus; namely to
provide innovative services to support their lives.

\section{Problem Statement}

In this section, I will discuss two real life scenarios which happens quite
often students and professors.

\subsection{Scenario 1: Hungry Student}
\label{HungryStudent}
XXX is a second year student, he has a lot of friend, he spends most of his day
at the University. His classes usually ends at 13.00 then he always have to
wait for his friends. Eventually when he reaches the canteen to eat with his
friends, he always finds long queue of students waiting to take food.
Furthermore he would prefer to know the menu of the canteen before standing in
the queue but he has to wait because of the huge gathering of students in queue.
Even when he reaches the food counter it takes a long time to prepare the food
he ordered. He is also not happy with the decreasing quality of the food in the
canteen. He also finds the canteen a  boring place to sit and chat with friends
as there is always huge gathering of students during the lunch break.

\subsection{Scenario 2: Busy Professor}
\label{BusyProfessor}
Mr. YYY is a professor of Computer Science who is busy with a lot of things like
research work, teaching in class, presentation talk, meetings. He does not have
enough time to choose food standing in the queue of the canteen. So he is
thinking if there will be a system where user can choose and order their lunch
even before going to the canteen, it would be even better if the system can
suggest dietary as well. If he can access the system through his mobile phone
from anywhere around the university it would make him a really happy professor.
\subsection{Real Life Problem}
In the previous section [\ref{HungryStudent},\ref{BusyProfessor}], I have
discussed two real life scenarios where technological service will make our life
more easy. Providing good facilities for eating can be a simple way to improve
campus life by the university cafeteria. Standing in long queues in cafeteria
for food takes out a good amount of time from student's and staff's
daily working routine. Additionally waiting in the queue without knowing the
menu for a particular day can be time killing if someone finds nothing
satisfactory to eat. The facility of knowing the menus in advance is a recurring
request. The canteen is a space for students to pass some quality time eating
and discussing with their friends similar for that university staff's with their
colleague's.

Daily and weekly menu are not available at cafeteria in advance and there is no
way to know cafeteria is open or not before going to the cafeteria physically.
It is always important to maintain proper diet from dieting suggestions
to keep healthy life. Healthy and happy life always has a great impact on daily
works such as study, research, job.

In this research, I have tried to figure out the possible answeres and solutions
of following questions:
\textbf{
\begin{enumerate}[(I)]
  \item How to skip the long queue of cafeteria such that no waiting more than 5 minutes to
  take food.
  \item How to know the menu of the cafeteria before standing in the queue;
  namely from any another places.
  \item How technology can help us in selecting meal from cafeteria more comfortably.
  \item How to know the most appropriate menu for me in terms of calorie contains and price.
  \item How to collaborate and share my feeling with others when I am in cafeteria.
\end{enumerate}
}


\section{Objectives}
In this thesis work, I have applied user centered design methodology and found
some possible services of Smart Cafeteria. First of all, I have applied
PACT\footnote{\href{http://hci.ilikecake.ie/requirements/pact.htm}{PACT
Analysis}} analysis on Smart cafeteria and figured out the following factors
through this analysis.

\textit{\textbf{Places:}} Meeting places for students, researchers and professors; the
canteen, the queue.

\textit{\textbf{Activities:}} Finding food menu to eat, avoid queue, choosing
the meals, eating, choose diet menu.

\textit{\textbf{Context:}} The university life, The activity of eating.

\textit{\textbf{Technologies:}} Web 2.0, SmartPhone.

Finally there are some services which should be considered in case of smart
cafeteria. In the thesis work, I will try to emphasize and develop such an
interactive, adaptive system to provide the following services:
% \begin{itemize}

\begin{enumerate}
%\addtocounter{enumi}{-1}
\item  \textit{\textbf{Mensa Queue Skipper:}} A system of booking food menu from
the canteen that makes the students wait no more than 5 minutes.
Queues in the canteen and at the administration services results loosing time.
Students often feel that the University cafeteria system is not enough to
provide them good supports for their daily lunch activities. This is due to
large number of students with respect to the number of personnel's providing
services in cafeteria. Technology could easily help them avoid queue. If all
member of the university could be able to access everydays food menu as a list
or search for their preferable menu with the help of an application as well as
ordering food and paying for the food through the application, It could be a
solution to skip queue in the cafeteria.

\item \textit{\textbf{Menu Finder:}} Daily and weekly menus of the University of
Trento cafeteria' are available on the website which is offered by cafeteria's
stuff in a specific format. But most of university's members do not notice that
information and the system does not notify them. So it is obvious that
university's cafeteria must need a menu finder to help them to search their
appropriate food menu which is offered by the cafeteria.

\item \textit{\textbf{Menu Suggester and Dieting Adviser:}} Students' dieting is
a very important issue that should not be left to chance. Some research
\cite{HealthyDiet} shows that the students' recurrent lack of information about
dieting, proper food combining meal plans, balance dieting and this could be an
obstacle to improve their general health and well being. So indications of
dieting measures in a cafeteria system such as Menu Suggester and Dieting
adviser depend on users' choice and how much calories they consumed in the last
couple of weeks could be a very good solution for Smart Cafeteria.

\item \textit{\textbf{Create Customized Menu:}} People can order food and make their
own menu in the restaurant according to their own choice and as much as food
they need. In the system application point of view, giving the user freedom to
create their own menu from different food could be a very smart thinking.

\item \textit{\textbf{Lunch with Friend:}} Create a collaborative application
where user can follow their friends; can see the activities of friends and share
the activity of meals. This may give newcomers and foreign students the
possibility to exchange language or psychological peer supporting.

\end{enumerate}

%\end{itemize}

The goal of the thesis is to develop just in time service where students,
professors, rersearchers do not need to wait too long to order food, we will
imposed such a system where students and staffs will be able to order their food
online even before going to the cafeteria physically. Daily and weekly menu will
be made available in the system which will also give suggestions of dieting
measures. The system will also suggest food according for their choice and the
calorie requirement. So the main objectives of this thesis are to:
\textbf{
\begin{enumerate}[(I)]
\item Create ``Smart Cafeteria'' which will be supported by web 2.0 system and
Smartphone application.
\item The application should be adaptive and interactive.
\end{enumerate}
}
There is also couple of sub goals which overlap with the main objectives of this
thesis. The sub goals are:
\textbf{ \begin{enumerate}[(i)] 
\item Provide online cafeteria services.
\item Provide dieting services to the students.
\item  Provide social collaboration services in the application.
\end{enumerate}
}

\section{Structure of Thesis}
The thesis is organized as follows:

% Chapter~\ref{chap:Introduction}, \textit{Introduction}, describes the set of
% scenarios, defines the problem statement and state the contribution of this
% work. This chapter also includes the organization of the thesis reflexively.

Chapter~\ref{chap:StateofArt}, \textit{State of Art}, highlights Human Computer
Interaction (HCI), its goals and adaptive HCI. It is also described Interaction
Design and its Process, Goals, Principles and Usability Heuristics. In the last
two sections of this chapter are discussed Interactive Mobile HCI and Benchmark
Analysis using some related mobile applications.

Chapter~\ref{chap:AnalysisofSmartCafeteria}, \textit{Analysis of Smart
Cafeteria}, discusses Requirements and Data Gathering techniques and processes
of ``Smart Cafeteria'' which help to find Stakeholders, initial Functional
Requirements and Non Functional Requirements as well as stable requirement using
Studying Document, Focus Group and Questionnaires. Finally Data Analysis has
been done using UML 2.0 such as use cases, class diagram and activity diagrams.

Chapter~\ref{chap:DesignofSmartCafeteria}, \textit{Smart Cafeteria Design},
presents Conceptual Architecture and Design, Mid-fadelity UI for both desktop
and Mobile version of Smart Cafeteria; and finally discusses Key Features of
the application as a mobile application.

Chapter~\ref{chap:EvaluationSC}, \textit{Usability Evaluation of Smart
Cafeteria}, reports Evaluation Methodology and Evaluation Result such as user
observation and Questionnaire for usability evaluation for both desktop and
mobile prototype. Finally Suggestions for Improvement has been presented.

Chapter~\ref{chap:Conclusion}, \textit{Conclusion}, concludes the overall
achievements of the thesis work and possible extensions of the work are being
presented.

