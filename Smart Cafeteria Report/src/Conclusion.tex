\chapter{Conclusion}
\label{chap:Conclusion}
In this thesis, it has been proposed an adaptive and interactive mobile
application for Smart cafeteria which gives very essential services to students
at everyday life in university. The main objectives were (I) Create ``Smart
Cafeteria'' which will be supported by web 2.0 systems and Smartphone
application. (II) The application should be adaptive and interactive. These
objectives has been achieved by finding stakeholders, functional requirements;
developing desktop, mobile prototype and usability evaluation using HCI
interaction design methodology for ``Smart Cafeteria''.


In depth, there was some sub goals of ``Smart Cafeteria'' which overlapped with
the main objectives and those were (i) Provide online cafeteria services, (ii)
Provide dieting services to the students, and (iii) Provide social collaboration
services into the system application. To achieve those objectives and sub goals,
I have analyzed requirements, developed prototypes for desktop and mobile
related to proposed services (i) Mensa Queue Skipper, (ii) Menu Finder, (iii)
Menu Suggester and Dieting Adviser, (iv) Create Customized Menu and (v) Lunch
with Friend.


The online cafeteria services are able to skip long queue in the cafeteria and
users' are able to browse, choose and order meal in advance before entering
cafeteria. In addition, users also able to know time schedule of cafeteria using
this application. Dieting services provides some functionalities such as
generate dieting report for users, suggesting food menu based on users choice
and having calories everyday at cafeteria, generates offers and notification
through email and sms which will help student keep happy and healthy life. In
the social collaboration, users shares meal activities, status with friends.


In this thesis, ``Smart Cafeteria'' has been analyzed, proposed the application,
designed both UI prototype for desktop and mobile, and finally usability
evaluation has been performed by through users' study. From the users'
evaluation, it is proven that it is essential providing quality services in
cafeteria for achieving the goal of a smart campus. By this proposed cafeteria
application's ideas and functionalities, the goals of ``Smart Cafeteria'' have
been achieved.


\section{Further Work}
\label{sec:FurtherWork}
Since this is the first time analysis and designs such an application in
cafeteria domain to provide services through mobile as well as browser to the
university students, it is very important to design high fidelity prototype,
programming model such as ORM model, controller, view and service oriented
architecture (SOA) to provide real services to users.


This moment, I have developed UML model and UI of application to demonstrate
``Smart Cafeteria'' and evaluated user experience following interactive design
methodology.  In future I will extend the work and build high fidelity
prototype, develop mobile services and perform users study to evaluate the
system again until maximum level of usability. I will do also more work with
application's adaptation using various machines learning approaches and figure
out which will be the best for ``Smart Cafeteria''. Finally I will develop rest of
the parts as full functional application of ``Smart Cafeteria'' which will ensure
more interactive, adaptability and usability.

